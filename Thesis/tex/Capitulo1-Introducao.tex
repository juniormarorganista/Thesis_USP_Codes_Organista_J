% ----------------------------------------------------------
%% Capitulo1-Introducao.tex
% ----------------------------------------------------------
% Introducao 
% ----------------------------------------------------------
\chapter[Introdução]{Introdução}
\label{Cap_Introducao}

Em diversas aplicações industriais, a dinâmica dos fluidos desempenha um papel central na eficiência e qualidade dos processos. Produtos industriais que envolvem o escoamento de fluidos, como o processamento de polímeros, a injeção de plásticos e a extração de petróleo, frequentemente dependem de fluidos com propriedades complexas, como os fluidos viscoelásticos. Esses fluidos, que combinam características de fluidos viscosos e sólidos elásticos, representam um desafio na modelagem e simulação devido à sua resposta não linear ao estresse e à deformação. Essa complexidade torna essencial o uso de métodos numéricos avançados e de alta precisão para prever o comportamento desses fluidos em diferentes condições. A substituição de materiais poliméricos por alternativas mais eficientes é uma prática comum em setores industriais, como a indústria automotiva, que busca otimizar processos e reduzir custos. Um estudo de caso relevante na indústria automobilística demonstrou a importância de selecionar adequadamente materiais poliméricos com propriedades específicas para aplicações exigentes, como a resistência ao impacto em baixas temperaturas e a compatibilidade com os processos de fabricação existentes \cite{bissoto2006substituiccao}. Esse cenário destaca a crescente necessidade de compreender o comportamento de escoamentos viscoelásticos em condições variadas, garantindo que o desempenho mecânico dos polímeros seja mantido ao longo da vida útil dos produtos. Assim, a compreensão e o controle dos escoamentos viscoelásticos tornam-se cruciais para assegurar a qualidade e a durabilidade dos produtos derivados de polímeros em diversas indústrias.

Na literatura, encontram-se diversos estudos que abordam modelos constitutivos para fluidos viscoelásticos, como as equações diferenciais de Maxwell \cite{beris1987spectral,mompean97}, Oldroyd-B \cite{brasseur1998time,mompean97,phillips2002,pinho2003}, White-Metzner \cite{White63}, Giesekus \cite{giesekus1962, giesekus1982, giesekus1985}, Leonov \cite{Leonov76}, modelos FENE\sigla*{FENE}{\textit{Finite Extensible Nonlinear Elastic}} \cite{Bird80,Bird83,christiansen77,Stevenson71,Warner72}, PTT \cite{phan-thien77,pinho2003}\sigla*{PTT}{Phan-Thien-Tanner} e derivados, Pom-Pom \cite{Larson88} e os modelos integrais: Maxwell \cite{Kaye62} e K-BKZ\sigla*{K-BKZ}{Kaye-Bernstein-Kearsley-Zapas} \cite{luo88}. 

O modelo de Maxwell foi uma das primeiras abordagens para descrever a viscoelasticidade em fluidos, combinando as propriedades de um sólido elástico de Hooke com as de um fluido viscoso newtoniano. Já o modelo Oldroyd-B, derivado da teoria cinética para soluções poliméricas concentradas e fundidas \cite{bird_v1_1987}, representa a cadeia polimérica como duas esferas conectadas por uma mola, onde as esferas correspondem ao centro de massa do sistema e as molas simulam a elasticidade das macromoléculas. Esse modelo é particularmente eficaz para representar fluidos com elasticidade ideal, como os fluidos Boger.

O modelo reológico proposto por \citeonline{giesekus1982} também se fundamenta em abordagens moleculares, utilizando sistemas compostos por esferas e molas, em que a mola segue a lei de Hooke. Diferentemente do modelo Oldroyd-B, o modelo Giesekus incorpora um efeito de anisotropia na formulação da força de arrasto que atua sobre as esferas. Isso resulta em uma equação similar aos modelos UCM\sigla{UCM}{\textit{Upper Convected Maxwell}} e Oldroyd-B, mas com a inclusão de termos não lineares, decorrentes dos produtos envolvendo o tensor de tensões. Outro modelo amplamente aplicado em simulações numéricas de fluidos viscoelásticos é o proposto por \citeonline{phan-thien77}, conhecido como PTT, que tem suas bases na teoria de redes de soluções concentradas e polímeros fundidos, levando em conta a energia elástica dessas redes.

Essas simulações ainda tem a vantagem de poder fornecer resultados precisos que representam com fidelidade o comportamento desses fluidos em diferentes cenários de escoamento. Contudo, a complexidade desse problema reside nas equações constitutivas que governam os fluidos viscoelásticos, tornando seu tratamento computacional desafiador. Com o desenvolvimento da tecnologia computacional, o interesse por simulações numéricas dessas aplicações industriais cresceu, uma vez que métodos numéricos permitem a modelagem de escoamentos de fluidos viscoelásticos com um custo significativamente menor do que em relação aos experimentos laboratoriais, proporcionando resultados precisos e que capturam o comportamento dos fluidos viscoelásticos sob diferentes condições. No entanto, essas simulações enfrentam desafios consideráveis, pois as equações constitutivas que descrevem esses fluidos são complexas e difíceis de tratar computacionalmente. Nesse contexto, a Dinâmica dos Fluidos Computacional (CFD, do inglês: \textit{Computational Fluid Dynamics})\sigla*{CFD}{\textit{Computational Fluid Dynamics}} se destaca como uma ferramenta indispensável para garantir a confiabilidade dos resultados gerados pelos métodos numéricos, e assim é fundamental realizar processos de Verificação e Validação. Esses procedimentos asseguram a precisão e a credibilidade das soluções encontradas. Conforme indicado pelo o Comitê Americano de Aeronáutica e Astronáutica (AIAA, do inglês:\textit{American Institute of Aeronautics and Astronautics})\sigla*{AIAA}{\textit{American Institute of Aeronautics and Astronautics}}\cite{AIAA2002} em seu guia, elaborado em 1998, para realização da Verificação e da Validação de códigos e resultados obtidos. Segundo \citeonline{oberkampftech}, esses processos envolvem a aplicação de conceitos e práticas rigorosas que garantem a correta implementação e confiabilidade dos códigos computacionais utilizados.

Os conceitos de Verificação e Validação, embora distintos, são complementares e sinérgicos, como apontado por \citeonline{oberkampftech}. A execução de um não elimina a necessidade do outro. De acordo com \citeonline{roacheartigo}, a Validação busca avaliar a precisão de uma simulação ao compará-la com dados experimentais disponíveis na literatura para o caso estudado. Já a Verificação, conforme também destacado por \citeonline{roacheartigo}, concentra-se em verificar se o código numérico executa corretamente os modelos implementados, avaliando a ordem de precisão e os erros numéricos com base em comparações com soluções conhecidas. Diferente da Validação, a Verificação não se preocupa com a correspondência entre o modelo e o mundo real, mas sim com a exatidão do processo de cálculo. Embora este trabalho reconheça a importância da Validação, será focado exclusivamente em testes de Verificação. Para mais detalhes sobre o processo de Validação, o leitor pode consultar as obras de \cite{roacheartigo} e \cite{oberkampftech}.

A verificação de código é fundamental na modelagem computacional, pois permite avaliar a precisão e a confiabilidade dos códigos numéricos aplicados na simulação de fenômenos físicos complexos \cite{Khoshghalb2019, Tranquilli2022}. Esse processo consiste em comparar os resultados obtidos pelo código com soluções analíticas conhecidas ou casos de referência que já possuem soluções estabelecidas \cite{Fernandes2018, Pedro2020}. Ao realizar a verificação, os pesquisadores garantem que os códigos estão corretamente implementados e são capazes de gerar resultados precisos. Esse processo é crucial para assegurar que o código capture de forma adequada a física do problema em análise. Além disso, a verificação permite identificar e corrigir possíveis erros ou inconsistências na implementação, garantindo que as simulações numéricas resultem em dados confiáveis e coerentes. Também auxilia na detecção das limitações dos métodos numéricos empregados, orientando o desenvolvimento de algoritmos mais robustos e precisos.

O Método das Soluções Manufaturadas (MMS, do inglês: \textit{Method of Manufactured Solution})\sigla*{MMS}{\textit{Method of Manufactured Solution}} tem um papel fundamental na verificação e validação de códigos computacionais voltados para a simulação de escoamentos de fluidos viscoelásticos \cite{Garcia2022}. Devido ao comportamento reológico complexo desses fluidos, a simulação numérica precisa apresenta desafios significativos \cite{Fernandes2017}. Em seu estudo, \citeonline{Tseng2021} investigaram o transiente de sobretensão em derretimentos poliméricos sob escoamento de cisalhamento inicial, analisando a eficiência da equação constitutiva de White-Metzner para captar variações expressivas na sobretensão em números elevados de Weissenberg. Os resultados numéricos para o coeficiente de crescimento da tensão de cisalhamento, em diversas taxas de cisalhamento, mostraram uma boa concordância com dados experimentais, corroborando a precisão do modelo. Por sua vez, \citeonline{Pimenta2019} implementaram solucionadores acoplados para simulações de escoamentos viscoelásticos transientes e em regime permanente, com acionamento elétrico, dentro da abordagem de volumes finitos, concluindo que solucionadores acoplados proporcionam maior precisão em simulações transientes e permitem o uso de passos de tempo maiores sem divergência numérica.

Já \citeonline{Fernandes2022} desenvolveram um método acoplado totalmente implícito para a resolução de escoamentos laminares, incompressíveis, não isotérmicos e viscoelásticos, utilizando a estrutura do tensor de conformação-log para números elevados de Weissenberg. O algoritmo foi validado ao ser aplicado em escoamentos de fluidos viscoelásticos Oldroyd-B não isotérmicos em problemas de referência envolvendo uma geometria de contração abrupta planar 4:1, tanto bidimensional quanto axissimétrica. Um dos problemas de referência frequentemente utilizado para validar novos métodos numéricos é o escoamento em cavidade com tampa móvel, que consiste no movimento da parede superior de um canal quadrangular, gerando um escoamento 2D \cite{Shankar2000}. Utilizando essa configuração, \citeonline{Abuga2020} apresentaram tanto resultados de validação quanto novos dados de referência para escoamentos viscoelásticos governados pelo modelo constitutivo Rolie-Poly, utilizando um solucionador baseado em volumes finitos de segunda ordem no OpenFOAM. Além disso, \citeonline{Fernandes2019} aplicaram o escoamento em cavidade com tampa móvel para avaliar o desempenho de seu algoritmo totalmente acoplado de segunda ordem em escoamentos laminares e incompressíveis com malhas colocalizadas. Similarmente, \citeonline{Tomio2020} validaram seu esquema numérico recém-desenvolvido, que emprega uma abordagem de diferenças finitas totalmente implícita de segunda ordem para discretizar os termos de convecção e difusão das equações governantes, utilizando a configuração de escoamento em cavidade com tampa móvel.

Esquemas de discretização de alta ordem são fundamentais na modelagem computacional, oferecendo diversas vantagens em relação aos métodos tradicionais de primeira e segunda ordens. Tais esquemas proporcionam maior precisão, permitindo a obtenção de soluções mais confiáveis e exatas para problemas complexos. Eles são capazes de capturar com eficiência características de pequenas escalas e gradientes acentuados, além de reduzir erros numéricos relacionados à dissipação e dispersão \cite{Souza2005}. Por exemplo, \citeonline{Souza2005} avaliaram o desempenho de diferentes esquemas de diferenças finitas na simulação de ondas de instabilidade em um escoamento bidimensional incompressível, utilizando uma formulação baseada na função corrente-vorticidade. Quatro esquemas de discretização espacial foram comparados, incluindo métodos de segunda a sexta ordem, sendo a integração no tempo realizada por um esquema de Runge-Kutta de quarta ordem. A análise demonstrou a superioridade dos esquemas de alta ordem na captura de fenômenos de instabilidade no escoamento.

De forma semelhante, \citeonline{Cueto2006} propuseram uma abordagem inovadora utilizando aproximações de quadrados mínimos móveis (MLS, do inglês: \textit{Moving Least Squares})\sigla*{MLS}{\textit{Moving Least Squares}} para desenvolver discretizações de volumes finitos de alta ordem em malhas não estruturadas. Essa técnica foi aplicada às equações de águas rasas, destacando-se pela sua precisão e flexibilidade, especialmente na modelagem de fluxos viscosos. Além disso, \citeonline{Silva2010} conduziram um teste de verificação com um código de Simulação Numérica Direta (DNS, do inglês: \textit{Direct Numerical Simulation})\sigla*{DNS}{Direct Numerical Simulation}, que utilizava esquemas de alta ordem e o Método das Soluções Manufaturadas, focado na evolução de ondas de instabilidade em escoamento de Poiseuille plano. O estudo comparou esquemas compactos e não compactos de alta ordem, bem como métodos espectrais, demonstrando sua eficácia.

Em outro trabalho, \citeonline{Fadel2011} desenvolveram um esquema de alta ordem baseado em diferenças finitas para resolver as equações de Stokes e Navier-Stokes incompressíveis, aplicando estênceis de alta ordem para gradientes de pressão e condições de contorno variadas. O método foi testado em diversos problemas bidimensionais, como escoamentos de Stokes, cavidade com tampa móvel e degrau recuado, destacando-se pela precisão e robustez. Além disso, \citeonline{ZHANG2019634} introduziram um esquema de alta ordem para reinicialização do método de conjunto de nível, comprovando sua eficácia em simulações de dinâmicas de interface, como a ruptura de gotas suspensas.

Por fim, \citeonline{ALMUSHAIRA2021235} apresentaram um método numérico de alta ordem estável e eficiente para resolver equações de reação-difusão fracionárias. Utilizando um esquema compacto de quarta ordem e a técnica de transferência matricial, juntamente com a Transformada Rápida de Fourier (FFT, do inglês: \textit{Fast Fourier Transform})\sigla*{FFT}{\textit{Fast Fourier Transform}}, o estudo demonstrou o desempenho da abordagem em modelos complexos, como Fitzhugh-Nagumo, Allen-Cahn, e Schnakenberg, mostrando precisão e estabilidade para problemas de reação-difusão tridimensionais.

Este trabalho tem como principal objetivo desenvolver uma Solução Manufaturada para a verificação de um código numérico utilizado na simulação de escoamentos de fluidos viscoelásticos, com foco nos modelos constitutivos UCM, Oldroyd-B, Giesekus e LPTT\sigla{LPTT}{\textit{Linear Phan-Thien-Tanner}}. A implementação da MMS permitirá avaliar a precisão e a exatidão do código, assegurando que os termos fontes introduzidos nas equações governantes sejam corretamente incorporados e processados. Com a verificação bem-sucedida, será possível garantir que o código numérico está adequado para resolver com rigor as equações desses modelos viscoelásticos complexos. Uma vez finalizado o processo de verificação, os termos fontes adicionados durante os testes poderão ser removidos, restabelecendo o código à sua forma original, de modo que ele possa ser utilizado em outras simulações e pesquisas realizadas pelo grupo. Assim, o presente trabalho não só reforça a confiabilidade do código em uso, mas também contribui para o avanço nas simulações numéricas de escoamentos de fluidos viscoelásticos, assegurando a robustez necessária para futuros estudos.

\section{Organização do trabalho}
Esta tese está organizada da seguinte forma:

No Capítulo \ref{Cap_FormulacaoMatematica} apresenta-se as equações governantes para os escoamentos incompressíveis de fluidos viscoelásticos, abrangendo os modelos constitutivos UCM, Oldroyd-B, Giesekus e LPTT. Neste capítulo, também são discutidos os processos de adimensionalização dessas equações, destacando os parâmetros adimensionais críticos, como o número de Weissenberg, que influenciam o comportamento dos fluidos viscoelásticos e o número Reynolds. A adimensionalização é essencial para garantir a generalidade dos resultados e facilitar a análise comparativa entre os diferentes modelos.

No Capítulo \autoref{Cap_MetodosNumericos} abordam-se os métodos numéricos utilizados para a solução das equações governantes. São detalhados os esquemas de discretização espacial e temporal, como diferenças finitas e métodos implícitos, além dos algoritmos implementados para resolver os sistemas de equações que descrevem os escoamentos viscoelásticos. A precisão e eficiência dos métodos numéricos são discutidas neste contexto.

O Capítulo \ref{Cap_MetodosSolucoesManufaturadas} é dedicado ao Método das Soluções Manufaturadas utilizado para a verificação do código numérico. Este capítulo descreve em detalhes o procedimento de geração dos termos fontes, necessários para a implementação da MMS e explica como essa técnica é empregada para verificar a ordem de convergência do código em relação aos modelos UCM, Oldroyd-B, Giesekus e LPTT.

No Capítulo \ref{Cap_ResultadosNumericos} apresenta-se os resultados obtidos a partir da verificação do código utilizando a Solução Manufaturada. Os resultados são discutidos em termos de precisão numérica, erros relativos e eficiência do código ao tratar os diferentes modelos constitutivos. Além disso, são avaliadas as implicações dos testes realizados sobre a robustez do código e sua aplicabilidade em simulações futuras.

No Capítulo traz-se as conclusões deste trabalho, enfatizando a importância da verificação numérica para garantir a confiabilidade de simulações de escoamentos de fluidos viscoelásticos. O capítulo também sugere possíveis melhorias no código e direções para trabalhos futuros.

No Apêndice \ref{chapter:mathematica_wolfran}, são apresentados os códigos utilizados para a geração dos termos fontes, que foram empregados na Solução Manufaturada e no Apêndice \ref{chapter:fortran_77} são apresentados os códigos em Fortran 77 que são resultados do caso que utilizamos neste trabalho. Esses códigos servem como referência para futuras implementações e adaptações.