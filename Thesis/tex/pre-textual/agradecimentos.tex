% ---
% Agradecimentos
% ---
Primeiramente, agradeço a Deus, por me guiar e me dar a força necessária para enfrentar cada desafio desta jornada. Sua presença constante foi meu alicerce em todos os momentos.

Aos meus pais, Valdir Organista e Vera Marcia Pichiteli Organista, meu amor e gratidão eternos. Vocês são minha maior inspiração, sempre acreditando em mim e oferecendo amor incondicional. Esta conquista é tanto de vocês quanto minha, pois foram seus ensinamentos e apoio que me trouxeram até aqui.

Às minhas queridas irmãs, Franciele Camila Organista e Gessica Organista, obrigado por estarem sempre ao meu lado, compartilhando risos, lágrimas e palavras de encorajamento. Ter vocês comigo em cada passo desta caminhada fez toda a diferença.

Ao meu orientador, professor Leandro Franco de Souza, meu sincero agradecimento não só por sua orientação e ensinamentos, mas também por seu companheirismo e amizade. Sua dedicação foi fundamental para este trabalho, e sua confiança em mim contribuiu imensamente para minha formação pessoal. Com especial carinho, agradeço a Leonardo José Martinussi pela eficiência e por estar sempre pronto a ajudar no que fosse preciso, especialmente com o uso do cluster Euler. Seu apoio foi essencial, e sem ele, esta pesquisa teria enfrentado desafios ainda maiores.

Aos meus amigos, Aquisson Theyllon Gomes da Silva, por estar presente durante toda essa trajetória e me apoiar nos momentos mais difíceis; Welton Costa Lavercio, por me ajudar a recuperar o ânimo nos momentos mais árduos; e aos meus "irmãos que nunca tive", Mateus Tozo, Laison Furlan, Uebert Gonçalves Moreira e Tatielly de Jesus Costa pela amizade verdadeira e por estarem ao meu lado em momentos cruciais. Sou profundamente grato por cada um de vocês estarem presentes na minha vida.

Finalmente, a todas as pessoas que de alguma forma fizeram parte desta caminhada, meu mais sincero agradecimento. Suas palavras, gestos e crença em mim foram fundamentais para que eu pudesse chegar até aqui.

O presente trabalho foi realizado com apoio da Coordenação de Aperfeiçoamento de Pessoal de Nível Superior - Brasil (CAPES) - Código de Financiamento 001.

À Universidade de São Paulo (USP), ao Instituto de Ciências Matemáticas e de Computação (ICMC), ao Centro de Matemática Aplicada à Indústria (CeMEAI) financiado pela FAPESP (Número do Processo: 2013/070375-0), aos professores deste instituto, e a todos que contribuíram para a realização desta pesquisa, de forma direta ou indireta.
